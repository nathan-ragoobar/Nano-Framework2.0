\chapter{dev/data organization}
\hypertarget{md_OriginalStuff_2dev_2data_2README}{}\label{md_OriginalStuff_2dev_2data_2README}\index{dev/data organization@{dev/data organization}}
\label{md_OriginalStuff_2dev_2data_2README_autotoc_md94}%
\Hypertarget{md_OriginalStuff_2dev_2data_2README_autotoc_md94}%


The idea is that each dataset has a .py file here in the root of {\ttfamily dev/data}, and each dataset then creates a directory here, and writes and caches anything inside that directory. So for example\+:


\begin{DoxyItemize}
\item running {\ttfamily python tinystories.\+py} will create a directory {\ttfamily tinystories} with its .bin files inside it
\item running {\ttfamily python tinyshakespeare.\+py} will create a directory {\ttfamily tinyshakespeare} with its .bin files inside it
\end{DoxyItemize}

And so on. This way we can nicely organize multiple datasets here, share common utilities between them, and then point the .py/.c code in the root of the project accordingly to these. 